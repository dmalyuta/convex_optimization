
\nextchapter{Introduction to Optimization}

\nextsubchapter{Definition of an optimization problem}

A \textit{mathematical optimization problem} has the form

\begin{equation}\label{eq:opt_problem}
  \begin{array}{rl}
    \text{minimize} & f_0(x) \\ \text{subject to} & f_i(x)\le b_i,\quad i=1,\ldots,m
  \end{array}
\end{equation}

where:

\begin{itemize}
\item $x=(x_1,\ldots,x_n)\in\mathbb R^n$ is the \textit{optimization variable};
\item $f_0:\mathbb R^n\to\mathbb R$ is the \textit{objective function};
\item $f_i:\mathbb R^n\to \mathbb R$, $i=1,\ldots,m$ are the \textit{constraint
    functions};
\item $b_1,...,b_m$ are the limits (bounds) for the constraints.
\end{itemize}

\begin{Fact}
  You can understand (\ref{eq:optimization_problem}) as an abstraction of
  choosing a vector in $\Reals^n$ from a set of candidate choices. In this case,
  constraints $f_i(x)\le bi$ are ``requirements'' or ``specifications'' while
  $f_0(x)$ is the cost of choosing $x$.

  \begin{itemize}
  \item If $f_0(x)$ is the cost, then $-f_0(x)$ is then the ``utility''
  \end{itemize}
\end{Fact}

\begin{Definition}
  A vector $x^*$ is \textit{optimal} or is a \textit{solution} if

  \begin{equation*}
    \forall z\in\mathbb R^n\text{ satisfying the constraints,}\quad f_0(x^*)\le f_0(z).
  \end{equation*}
\end{Definition}

\nextsubchapter{Classes of optimization problems}

Optimization problems are bundled into classes based on the form of their
objective ($f_0$) and constraint ($f_i$) functions. Here are some below.

\begin{Fact}
  Note: in optimization jargon, a "program" is an alias for an "optimization
  problem".
\end{Fact}

\smalltitle{Least-Squares Problems}

\begin{Definition}
  A \textbf{least-squares problem} is an optimization problem with no
  constraints and an objective function which is affine in the optimization
  variable $x$. Let $A\in\Reals^{k\times n}$ (with $k\ge n$) and $a_i^T$ be its
  rows. Then the optimization problem is:
  \begin{equation*}
    \text{minimize} f_0(x) = \fnorm{Ax-b}{2}^2 = \sum_{i=1}^k (a_i^Tx-b_i)^2
  \end{equation*}
\end{Definition}

Least-squares programming is a \textit{mature technology}: least-squares
problems can be solved efficiently and reliably. Many software packages exist
that do this.

\begin{Fact}
  Least-squares properties:
  \begin{itemize}
  \item The analytical solution for $A\in\mathbb R^{k\times n}$ tall
    (i.e. $k>n$) and full rank (i.e. full column rank,
    $\text{null}(A)=\emptyset$): $x^*=(A^TA)^{-1}A^Tb$;
    \begin{itemize}
    \item NB: good solvers don't do literally this operation, though!
    \end{itemize}
  \item Compute time $\propto n^2k$.
    \begin{itemize}
    \item Where $n$ is ``\textit{small}'' while $k$ is ``\textit{large}''. Just
      remember this: computation time is ``small squared times large''.
    \end{itemize}
  \end{itemize}
\end{Fact}

How to know if your problem is a least-squares problem? \textbf{Simple, just one
  question}:

\begin{itemize}
\item Q: is the objective function the 2-norm squared of an affine function of
  $x$ (and you have no constraints)?
  \begin{itemize}
  \item A: Yes: it's a least-squares problem!
  \item A: No: it's not a least squares problem!
  \end{itemize}
\end{itemize}

Applications to least-squares programming include:

\begin{itemize}
\item Regression analysis
\item Parameter estimation
\item Data fitting methods
\item Optimal control
\end{itemize}

\begin{Fact}
  A variation of standard least squares is \textit{weighted least squares} with
  weights $w_i\ge 0$ $\forall i$ reflecting the relative concern for term
  $a_i^Tx-b_i$ being large:
  \begin{equation*}
    \text{minimize}\sum_{i=1}^kw_i(a_i^Tx-b_i)^2
  \end{equation*}
\end{Fact}

\begin{Fact}
  Another technique in least squares is \textit{regularization} which adds extra
  terms for the cost function, e.g.
  \begin{equation*}
    f_0(x) = \sum_{i=1}^k(a_i^Tx-b_i)^2+\rho\sum_{i=1}^k x_i^2
  \end{equation*}
  where $f_0(x)$ can still be formulated as a least squares problem. The extra
  terms penalize large values of $x$ and result in a sensible solution in cases
  where minimizing the first sum only does not. Parameter $\rho\in[0,1]$ trades
  off between meeting the original objective ($\rho=0$) and keeping $x_i^2$
  small ($\rho=1$).

  \begin{Example}
    Regularization is used in statistical estimation when $x$ is given a prior
    distribution.
  \end{Example}
\end{Fact}

\smalltitle{Linear program}

\begin{Definition}
  (\ref{eq:opt_problem}) is a \textbf{linear program} if $f_0,...,f_m$ are
  linear functions (i.e. both objective and constraints).
\end{Definition}

\begin{Definition}
  Recall that:
  $$
  f(x):\mathbb R^n\to\mathbb R\text{ is linear}\Leftrightarrow \forall
  x,y\in\mathbb R^n,\,\,\forall\alpha\in\mathbb \,\,\, f(\alpha x+\beta
  y)=\alpha f(x)+\beta f(y).
  $$
\end{Definition}

\textit{Linear programming}, or \textit{linear optimization}, is the act of
solving a linear program. A linear program may be more specifically formulated
as:

\begin{equation*}
  \begin{array}{rl}
    \text{minimize} & c^Tx \\ \text{subject to} & a_i^Tx\le b_i,\quad i=1,\ldots,m
  \end{array}
\end{equation*}

No simple analytical formula exists for the solution of a linear program!
However, like for least squares, very reliably and effective methods exist to
solve this type of problem. Like least squares, linear programming is a
\textit{mature technology}. Existing algorithms are (almost) as reliable as
least squares.

\begin{Fact}
  Some linear programming properties:
  
  \begin{itemize}
  \item No analytical solution (except for trivial cases)!
  \item Computation time $\propto n^2m$ if $m\ge n$ (more constraints than
    optimization variables);
    \begin{itemize}
    \item Less with structure;
    \item NB: this is the same computational time as least-squares!
    \item It is good news that it's $n^2m$ and not $m^2n$ (since generally there
      can be very many constraints on a relatively small optimization variable)
    \end{itemize}
  \end{itemize}
\end{Fact}

Unlike least squares problems, which are readily recognized by asking oneself
the ``is the objective function the 2-norm of an affine function'' question,
recognizing linear programs is more involved (i.e. it requires more skill from
the engineer). However, it is a skill that is readily acquired since only a few
standard tricks are involved (see \TODOREF).

\smalltitle{Convex program}

\begin{Definition}
  A \textbf{convex optimization problem} is one in which
  $f_0,\ldots,f_m$ are convex functions.
\end{Definition}

\begin{Definition}
  Function $f(x):\mathbb R^n\to\mathbb R$ is convex $\Leftrightarrow$
  \begin{equation*}
    \forall x,y\in\mathbb R^n, \forall \alpha,\beta\in\mathbb R_+\text{
      s.t. }\alpha+\beta=1\quad\Rightarrow\quad f(\alpha x+\beta y)\le \alpha f(x)+\beta f(y).
  \end{equation*}
\end{Definition}

Unlike general nonlinear programs, efficient and reliable algorithms exist for
solving convex programs (though perhaps not as reliable as linear or least
squares programs - but almost as reliable). Given also their wide applicability
to real world problems, this makes convex optimization ideally suited for real
world (and even high stakes) applications!

\begin{Fact}
  Linear and least squares programs are a subset of convex programs. In turn,
  convex programs are a subset of nonlinear programs.
\end{Fact}

\begin{Fact}
  Interior point methods can solve convex problem (\ref{eq:opt_problem}) in
  almost always 10 to 100 iterations. Ignoring any problem structure
  (e.g. sparsity), each iteration requires on the order of $\{n^3, n^2m, F\}$
  operations where $F$ is the cost of evaluating the first and seconds
  derivatives of $f_0,...,f_m$.
\end{Fact}

Recognizing a convex function \textbf{can be difficult}. There are many more
tricks involved than for linear programming! Thus, recognizing that a problem is
a convex optimization problem, or that it can be cast into one, can be
challenging.

\begin{Fact}
  The main goal of this document is to learn to recognize and formulate convex
  optimization problems! This is an essential skill because the challenge and
  the art of using convex optimization is in recognizing and formulating the
  problem - because solving it is (almost) a mature technology.
\end{Fact}

\smalltitle{Nonlinear program}

\begin{Definition}
  \textbf{Nonlinear optimization} (or \textbf{nonlinear programming}) describes
  optimization problems where $f_0,...,f_m$ are not linear, but are also no
  tknow to be convex.
\end{Definition}

Nonlinear optimization can be applied to basically any optimization problem,
however:

\begin{itemize}
\item It may take a very long time to solve;
\item It may not find the (globally optimal) solution - i.e. you may find
  $\bar x$ s.t. $f(\bar x) > f(x^*)$, and will never know of the $x^*$ ``out
  there'';
\item It may be very difficult to solve!
\end{itemize}

\begin{Example}
  \textbf{Local optimization}. This approach gives up finding the solution $x^*$
  and resorts to finding a locally optimal solution, which minimizes $f_0(x)$
  among feasible points near an initial value $x_0$.
  \begin{itemize}
  \item Local optimization can be fast and can handle large-scale problems and
    requires only that $f_0,...,f_m$ are differentiabile;
  \item Local optimization is valuable in fields where there is value in finding
    a good $x$, if not the best one. Example: engineering design, where $x_0$ is
    the manual initial design;
  \item It requires setting an initial point, and the solution it finds can
    greatly vary with the $x_0$ that you choose!  Not good for autonomous
    applications...
  \end{itemize}

  Roughly speaking, \textit{local optimization is more art (although well
    developed art) than technology}.
\end{Example}

\begin{Example}
  \textbf{Global optimization}. This approach aims to find the true solution
  $x^*$, however compromises efficiency. Worst-case complexity: exponential
  growth with $n$ (size of optimization variable) and $m$ (number of
  constraints).

  \begin{Fact}
    Think: ``curse of dimensionality'' for dynamic programming, which tries to
    find the globally optimal policy but which in pratice cannot be applied to
    control problems with much more than 5 state variables!
  \end{Fact}

  Global optimization methods are applied in domains where computing time is not
  critical and the value of finding the true global solution is very high, such
  as \textit{worst-case analysis} or \textit{verification} of high value/safety
  critical systems.
  
\end{Example}

\begin{Fact}
  Almost every method for global optimization is based on convex optimization as
  a subroutine. Convex optimization is found nonlinear optimization in the
  following applications:
  \begin{itemize}
  \item Initialization for location optimization (by solving a convex
    approximation of the true nonlinear problem and using the solution as the
    initial point for the local optimization routine)
  \item Convex heuristics for nonlinear optimization
  \item Finding sampling probability distribution parameters for
    \textit{randomized algorithms} that minimizes expected value of $f_0$
  \item Finding bounds for global optimization (via \textit{relaxation} and
    \textit{Lagrangian relaxation})
  \end{itemize}
\end{Fact}




%%% Local Variables:
%%% mode: latex
%%% TeX-master: "document"
%%% End:
