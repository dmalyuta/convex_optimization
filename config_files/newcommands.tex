\newcommand*{\QED}{\ensuremath{\blacksquare}}
\newcommand{\union}{\cup}
\newcommand{\aff}[1]{\textbf{aff }#1}
\newcommand{\relint}[1]{\textbf{relint }#1}
\newcommand{\cl}[1]{\textbf{cl }#1}
\newcommand{\conv}[1]{\textbf{conv }#1}
\newcommand{\intersection}{\cap}
\newcommand{\overeq}[1]{\stackrel{\mathclap{\tiny{\substack{#1}}}}{=}}
\newcommand{\overto}[1]{\stackrel{\mathclap{\tiny{\substack{#1}}}}{\to}}
\newcommand{\linA}{\mathcal{A}}
\newcommand{\linB}{\mathcal{B}}
\newcommand{\abs}[1]{|#1|}
\newcommand{\wpw}{\omega_{\text{pw}}}
\newcommand{\linC}{\mathcal{C}}
\newcommand{\SI}{S_{\text{I}}}
\newcommand{\TI}{T_{\text{I}}}
\newcommand{\SO}{S_{\text{O}}}
\newcommand{\TO}{T_{\text{O}}}
\newcommand{\LI}{L_{\text{I}}}
\newcommand{\LO}{L_{\text{O}}}
\newcommand{\Hinf}{\mathcal{H}_\infty}
\newcommand{\Htwo}{\mathcal{H}_2}
\newcommand{\perpoplus}{\overset{\perp}{\oplus}}
\newcommand{\inv}{^{-1}}
\newcommand{\GM}{\text{GM}}
\newcommand{\PM}{\text{PM}}
\newcommand{\TV}{\text{TV}}
\newcommand{\fnorm}[2]{\left\|#1\right\|_{#2}}
\newcommand{\Property}[1]{
	\begin{properties}
		\item[\textit{Property #1}.]
	\end{properties}
}
\def\AR{\text{\itshape\clipbox{0pt 0pt .32em 0pt}\AE\kern-.30emR}}
\newcommand{\skewmat}[1]{\bm{#1}^{\times}}
\newcommand{\proj}[2]{\text{proj}_{#1}(#2)}
\newcommand{\tabitem}{~~\llap{\textbullet}~~}
\newcommand{\basis}[3]{\{#1_{#2}\}_{#2=1}^{#3}}
\newcommand{\sequence}[3]{\{#1_{#2}\}_{#2=0}^{#3}}
\newcommand{\Naturals}{\textbf{N}}
\newcommand{\Reals}{\textbf{R}}
\newcommand{\Complex}{\textbf{C}}
\newcommand{\argmin}{\operatornamewithlimits{argmin}}
\newcommand{\argmax}{\operatornamewithlimits{argmax}}
\newcommand{\then}{\Rightarrow}
\newcommand{\Rank}{\text{rank}}
\newcommand{\Range}{\text{range}}
\newcommand{\Nullity}{\text{nullity}}
\newcommand{\Ker}{\text{ker}}
\newcommand{\Null}{\text{null}}
\newcommand{\Det}{\text{det}}
\newcommand{\Spec}{\text{spec}}
\newcommand{\DIM}{\text{dim}}
\newcommand*\bigcdot{\mathpalette\bigcdot@{.5}}
\newcommand{\InnerProd}[2]{\langle#1,#2\rangle}
\newcommand{\scaleval}{0.8}
\newcommand{\Expectation}[1]{\textbf{E}}
\newcommand{\Variance}[1]{\text{Var}[#1]}
\newcommand{\Cov}[2]{\text{Cov}[#1,#2]}
\newcommand{\Prob}{\textbf{prob}}
\newcommand{\ord}[1]{^{\left(#1\right)}}
\DeclarePairedDelimiter\ceil{\lceil}{\rceil}
\DeclarePairedDelimiter\floor{\lfloor}{\rfloor}
\newcommand{\mc}[2]{{\color{#1}#2}}
\definecolor{mygrey}{RGB}{51, 51, 51}
\newcommand{\TODOREF}{\textbf{TODO\_REF}}

%%%%%%%%%% Drawings
\newcommand*\circled[1]{\tikz[baseline=(char.base)]{\node[shape=circle,draw,inner sep=0pt,minimum size=0em] (char) {#1};}}

\newcommand*\circledsmall[1]{\tikz[baseline=(char.base)]{\node[shape=circle,draw,inner sep=0pt,minimum size=1em] (char) {#1};}}

\newcommand*\circledcolor[2]{\tikz[baseline=(char.base)]{\node[shape=circle,draw,inner sep=0pt,minimum size=1.3em,fill=#2] (char) {#1};}}

\newcommand*\squaredcolor[2]{\tikz[baseline=(char.base)]{\node[shape=rectangle,draw,inner sep=0.5pt,minimum size=0.1em,fill=#2] (char) {#1};}}

\newcommand*\squaredcolornobord[2]{\tikz[baseline=(char.base)]{\node[shape=rectangle,inner sep=0.5pt,minimum size=0.1em,fill=#2] (char) {#1};}}

\newcommand*\squaredcolornobordsetsep[3]{\tikz[baseline=(char.base)]{\node[shape=rectangle,inner sep=#3,minimum size=0.1em,fill=#2] (char) {#1};}}

\newcommand\dangersign[1][1.5em]{%
  \renewcommand\stacktype{L}%
  \scaleto{\stackon[0.7pt]{\color{red}$\triangle$}{\tiny !}}{#1}%
}

%%%%%%%%%% Big commands
\newcommand{\distribdiscript}[6]{
\noindent\rule[0ex]{\linewidth}{0.5pt}
\flushleft{\textbf{#1 ($X\sim #2$)}} % Name (#1) and notation (#2)
\vspace{-2mm}
\noindent\rule[1.5ex]{\linewidth}{0.5pt}
#3 % Description of distribution
\begin{tabularx}{1\columnwidth}{X|X}
\hline
#4 % PMF/PDF/CDF go here
&
Properties:
#5 % Properties of distribution go here
\\ \hline
\end{tabularx}
\myfigure{
\includegraphics[width=0.9\columnwidth]{#6.pdf} % Filename of the distribution grpahic goes here
}
}

\newcommand{\newfigure}[4]{
\begin{figure}[H]
\centerline{\includegraphics[width=#1\columnwidth]{figures/#2}}
\caption{#3}
\label{#4}
\end{figure}
\vspace{0}
}

\newcommand{\newfigureNocaption}[2]{
\begin{center}
\includegraphics[width=#1\columnwidth]{figures/#2}
\end{center}
}

\newcommand{\nextchapter}[1]{
\begin{Chapter}
	\section{#1}%\section{}
\end{Chapter}
}

\newcommand{\nextchapterhidden}[1]{
\begin{Chapter}
	\section*{#1}
\end{Chapter}
}

\newcommand{\nextsubchapter}[1]{
\begin{Subchapter}
	\subsection{#1}
\end{Subchapter}
}

\newcommand{\nextappendix}[1]{
\begin{Appendix}
	\section{\normalfont{\textit{#1}}}
\end{Appendix}
}

\newcommand{\myfigure}[1]{
\vspace{-4mm}
\begin{figure}[H]
\centering
#1
\end{figure}
\vspace{-4mm}
}

\newcommand{\separator}{
\begin{center}
\vspace{-3.5mm}
\noindent\rule{1\columnwidth}{0.3pt}
\vspace{-3.5mm}
\end{center}
}

\newcommand{\separatorfat}{
\begin{center}
\vspace{-1.5mm}
\noindent\rule{1\columnwidth}{0.4pt}
\vspace{-2.5mm}
\end{center}
}

\newcommand{\StateSpace}[4]{
\left[
\begin{array}{c|c}
#1 & #2 \\ \hline
#3 & #4
\end{array}
\right]
}

\newcommand{\expm}[1]{\mathrm{e}^{#1}}

\newcommand{\smalltitle}[1]{
\begin{tabularx}{\columnwidth}{|X|}
	\hline
	\textbf{#1} \\
	\hline
\end{tabularx}	
\hspace{-2mm}

}
